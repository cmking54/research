\documentclass[../exps.tex]{subfiles}
%\documentclass[letterpaper,11pt]{article}
\begin{document}
\pagestyle{empty}
\noindent{\Large{\textbf{Experiment 2}}
} 

\noindent{\large{Gathering Data And Measuring Ratios Found}} \\

\noindent{\underline{Purpose:}} 

To give some validity to the premise of a golden ratio in humans in reality
\\

\noindent{\underline{(Mini)-Hypothesis:}} 

If the golden ratio exists as documented, it should not diverge too far what is reported.
\\

\noindent{\underline{Method:}}

As the algorithm runs, it can record a running average (or the data itself) and that can be compared to the ratio given by sources on the idea of a golden ratio. Also it should performed on different subjects, as it will add diversity.
\\

\noindent{\underline{Prediction:}}

There will be a ratio either way; whether it lines up to Vitruvius or not. That ratio finding will build a mathematical model of what dimensions make up the user. If they are unique enough from the golden ratio and others, it could be used a identifier for the particular user. If it is does fall in line with the golden ratio, then it can show the validity to the golden ratio.
\\
\end{document}
